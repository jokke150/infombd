\documentclass[../essay.tex]{subfiles}
\begin{document}
The discovery of \textbf{frequent itemsets} is one of the major families of
techniques for characterizing data.
To explain how frequent itemsets work, we will take a look at the
\emph{market-basket} model first.
The market-basket model and frequent itemsets problem originated in the analysis
of true market baskets.
Retailers wanted to learn what items are frequently bought together.
In the \emph{market-basket} model we look at the many-to-many relationship
between the \emph{items} and the \emph{baskets}.
The \emph{frequent itemsets} problem follows from this model.
We are concerned with finding the sets of items that appear in many of the same
baskets.
To give a more formal definition: we have a number $s$, which is known as the
\emph{support threshold}.
The \emph{support} for a set of items $I$ is the number of baskets for which $I$
is a subset.
We say that a set of items $I$ is frequent if its supports is $s$ or more.

\textbf{Association Rules}
\end{document}
