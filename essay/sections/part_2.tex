\documentclass[../main.tex]{subfiles}
\begin{document}

%% The second part compares Toivonen's bounds and d-bound sample size by performing experiments
%% on frequent itemsets.

\section*{Part II}
\setcounter{section}{2}
\setcounter{subsection}{0}
\addcontentsline{toc}{section}{Part II}

\subsection{Introduction}
\label{sec:II_intro}

In the second part of the essay we will discuss the results of an experimental study performed on mining of frequent itemsets.
The goal of the experiments is to compare Toivonen's bounds to the d-bound.
How big of a sample do each of the bounds prescribe, and how good are the approximate results.
Both bounds use the parameter pair $(\epsilon, \delta)$, which we will vary accordingly.
In addition to the parameter pair, Toivonen's bounds also uses $\mu$.

Based on \emph{Toivonen's} bounds we will need a minimal sample size of

\begin{tabularx}{\textwidth}{X X}
    {\begin{align}
        &\abs{S} \ge \frac{1}{2\epsilon^{2}} \ln \frac{2}{\delta}
    \end{align}}
     & 
    {\begin{align}
        &\sqrt{\frac{1}{2n}\ln\frac{1}{\mu}}
    \end{align}}
\end{tabularx}

Based on the \emph{d-bound} we will need a minimal sample size of
\begin{align}
    \abs{S} \ge \min \left\{ \abs{D}, \frac{4c}{\epsilon^{2}}\left( d + \log \frac{1}{\delta} \right) \right\}
\end{align}
Where $D$ is the data set.

\subsection{Experimental Set Up}
\label{sec:II_setup}

The experimental setup is based on the experiments described in the works by \citeauthor{Riondato2015} \cite{Riondato2015},
the book \citetitle{Leskovec2014mining} \cite{Leskovec2014mining},
and the slides of the lectures given during the Big Data course at Utrecht University \cite{Siebes2017fim}.
The following steps describe the process used during the experiments:
\begin{enumerate}[noitemsep]
    \item Inputs: a data set $\mathcal{D}$, and a parameter set $(\epsilon, \delta, \mu)$.
    \item Compute the d-index of the data set $\mathcal{D}$.
    \item Based on the parameters, compute Toivonen's and the d-bound sample sized.
    \item \label{enum:sampling} Mine using sampling:
        \begin{enumerate}[noitemsep]
            \item Draw (with replacement) a sample of sufficient size.
            \item Compute the set $\mathcal{F}$ of frequent itemsets on this sample (using the lowered threshold).
            \item Check the support of the elements of $\mathcal{F}$ on the complete data set $\mathcal{D}$.
        \end{enumerate}
    \item Repeat step \ref{enum:sampling} $n$ times for each set of parameters --- we want to minimize the effect of the random sampling.
\end{enumerate}

We used data sets from the FIMI repository \cite{FIMI2003}.
Just as in the paper by \citeauthor{Riondato2015}, we use the 'blow up' trick to replicate the data set a number of times,
so that it is more representable compared to the sizes of modern data sets \cite[Section~5]{Riondato2015}.
The sizes in the table below reflect the sizes of the data sets after applying the 'blow up' trick.

\begin{center}
\begin{tabular}{l r r}
    \hline
    Name & Size & d-index \\
    \hline
    accidents & & 46 \\
    connect & & 43 \\
    kosarak & & 443 \\
    pumsb\_star & & 59 \\
    retail & & 58 \\
    \hline
\end{tabular}
\end{center}

The three parameters were varied accordingly:
$\delta = \left\{ 0.01, 0.05, 0.1 \right\}$,\\
$\epsilon = \left\{ 0.01, 0.015, 0.02, 0.025 \right\}$, and
$\mu = \left\{ \right\}$.

The experiments and mining of frequent itemsets were implemented in Python,
using \citeauthor{Borgelt2016}'s PyFIM library \cite{Borgelt2016}.
The implementation can be found on GitHub \cite{Robeer2017}.

\subsection{Results}
\label{sec:II_results}

\subsection{Conclusion}
\label{sec:II_conclusion}

\end{document}